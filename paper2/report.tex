\documentclass[sigconf]{acmart}


\input{format/i523}


\begin{document}
\title{Big Data Applications in Using Neural Networks for Medical Image Analysis}


\author{Tyler Peterson}
\orcid{1234-5678-9012}
\affiliation{%
  \institution{Indiana University - School of Informatics, Computing, and Engineering}
  \streetaddress{711 N. Park Avenue}
  \city{Bloomington} 
  \state{Indiana} 
  \postcode{47408}
}
\email{typeter@iu.edu}


% The default list of authors is too long for headers}
\renewcommand{\shortauthors}{G. v. Laszewski}


\begin{abstract}

 Medical image analysis is proving to be a promising domain for disruption by machine learning. The analysis of medical imagery has long been within the purview of radiologists, a specialization in medicine that reviews medical imaging to form diagnoses and advise on treatment options. Historically, radiologists have relied on their training, senses and years of experience to evaluate images for medical issues, such as the presence of tumors, lung nodules, and hip osteoarthritis. The presence of technology, generally referred to as computer-aided diagnosis (CAD) tools, has been growing over the last several decades, but modern computing power and sizable datasets has accelerated the effectiveness of these assistive tools. Machine learning algorithms, especially artifical neural networks (ANN), are being leveraged to help identify problems present in medical images at a great level of success. Several research studies conclude that ANN techniques can match, and occassionally outperform, the abilities of radiologists. Big data and the application of advanced algorithms show promise for evolving our ability to successfully evaluate medical images and save lives in the process.      
  
\end{abstract}


\keywords{i523, hid331, Big Data, Medical Image Analysis, Artificial Neural Networks, Medicine}


\maketitle


\section{Introduction}

 This is my introduction \cite{editor00}. The increased attention and effectiveness of these tools is brought on by several factors, including sufficient computing power and the availability of sufficiently large train sets needed by the ANN to understand patterns in the data.


\section{Conclusion}

 This is my conclusion.

 
\begin{acks}

 These are my acknowledgements

\end{acks}


\bibliographystyle{ACM-Reference-Format}
\bibliography{report} 


\appendix
 

\section{Issues}

\DONE{Example of done item: Once you fix an item, change TODO to DONE}

\subsection{Writing Errors}

    \TODO{Errors in title, e.g. capitalization - in}




\end{document}
