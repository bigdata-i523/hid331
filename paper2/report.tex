\documentclass[sigconf]{acmart}


\usepackage{graphicx}
\usepackage{hyperref}
\usepackage{todonotes}

\usepackage{endfloat}
\renewcommand{\efloatseparator}{\mbox{}} % no new page between figures

\usepackage{booktabs} % For formal tables

\settopmatter{printacmref=false} % Removes citation information below abstract
\renewcommand\footnotetextcopyrightpermission[1]{} % removes footnote with conference information in first column
\pagestyle{plain} % removes running headers

\newcommand{\TODO}[1]{\todo[inline]{#1}}


\begin{document}
\title{Big Data Applications in Using Neural Networks for Medical Image Analysis}


\author{Tyler Peterson}
\orcid{1234-5678-9012}
\affiliation{%
  \institution{Indiana University - School of Informatics, Computing, and Engineering}
  \streetaddress{711 N. Park Avenue}
  \city{Bloomington} 
  \state{Indiana} 
  \postcode{47408}
}
\email{typeter@iu.edu}


% The default list of authors is too long for headers}
\renewcommand{\shortauthors}{G. v. Laszewski}


\begin{abstract}

  Medical image analysis is proving to be a promising domain for disruption by machine learning. The analysis of medical imagery has long been within the purview of radiologists, a specialization in medicine that reviews medical imaging to form diagnoses and advise on treatment options. Historically, radiologists have relied on their training, senses and years of experience to evaluate images for medical issues, such as the presence of tumors, lung nodules, and hip osteoarthritis. The presence of technology, generally referred to as computer-aided diagnosis (CAD) tools, has been growing over the last several decades, but modern computing power and sizable datasets have accelerated the effectiveness of these tools. Machine learning algorithms, especially artifical neural networks (ANN), are being leveraged to help identify abnormalities present in medical images at a high level of accuracy. Several research studies conclude that ANN techniques can match, and occassionally outperform, the abilities of radiologists. Big data and the application of advanced algorithms show promise for evolving our ability to successfully evaluate medical images and save lives in the process.

\end{abstract}


\keywords{i523, hid331, Big Data, Medical Image Analysis, Artificial Neural Networks, Medicine}


\maketitle


\section{Introduction}

The analysis of medical imagery is primarily the responsibility of radiologists. These individuals are medical doctors who specialize in diagnosing diseases through review of images produced by various imaging modalities, such as x-ray, ultrasound, computerized tomography (CT), magnetic resonance imaging (MRI) and positron emission tomography (PET) \cite{cite00}. Radiologists serve as an expert to other physicians by analyzing the medical images of patients suspected of having certain medical issues, and by making recommendations on subsequent care based on the observations \cite{cite00}. The images reviewed by radiologists are generally stored digitially, and images are increasingly being stored in picture archiving and communications systems (PACS). These systems are required to keep up with the rapid accumulation of medical image data. Between 2005 and 2011, the medical image data in US hospitals increased from only 8,900 terabytes to 27,000 terabytes \cite{cite04}. That number is expected to grow by 20 percent every year due to increasing image size and resolution, the adoption of 3D imaging, and an aging population who will likely bring an increasing demand for medical imaging studies \cite{cite04}.

It is estimated that one billion medical images are created worldwide each year, and most of these are assessed by radiologists \cite{cite01}. Given that radiologists are human, their judgment is fallible. It is estimated that the lowest average error rate in analyzing medical imagery is 4 percent, which means collectively radiologists are estimated to make 40 million errors in judgement every year \cite{cite01}. A particularly striking example of fallibility comes from a study that analyzed the first and second interpretations of radiologists from Massachusetts General Hospital. They reviewed abdominal CTs and re-reviewed studies that had either been interpreted by themselves or a colleague. The study found that the radiologists disagreed with their peers 30 percent of the time and even disagreed with themselves 25 percent of the time \cite{cite01}.

There are two major types of radiologic analysis error: perceptual error and interpretive errors \cite{cite01}. Most errors, up to 80 percent, are perceptual errors, which occur when an abnormality is not perceived by the reviewer during the initial review, but is identified in a subsequent analysis \cite{cite01}. Interpretive errors occur when the radiologist successfully identifies the abnormality, but incorrectly diagnosis the problem, which may lead to a less appropriate course of careaction \cite{cite01}. There are several reasons why errors occur, including fatigue, excessive pace of analysis, distractions and insufficient knowledge of the practitioner. It is also asserted that the extreme complexity of a radiologist's job contributes to the errors. Errors occur in the practice of radiologists all across the world, at varying levels of training, in all imaging modalities and all clinical settings \cite{cite01}.

Over the last several decades, there has been an effort to develop computer-based tools to aid radiologists in the detection of abnormalities. Computer-aided diagnosis (CAD) tools are primarily intended to increase the rate at which problems are identified while also reducing the false negatives resulting from human error \cite{cite06} . These systems are intended to supplement, not replace, the radiologist by reporting a second opinion to be considered alongside the radiologist's assessment. The earliest initiatives to develop these tools occurred in the 1960s, and concerted efforts began in the 1980s \cite{cite02}. Despite the research being nearly 60 years old, widespread adoption is a relatively recent occurrence \cite{cite03}. Clinical studies reported early CAD implementations as being minimally effective. Specifically, CAD decisions included more false positives than human assessments, which created more work for radiologists and often led to additional, unnecessary medical tests and biopsies \cite{cite05}.

Several improvements in the field of computing have increased the accuracy of CAD tools and subsequently encouraged wider adoption of these tools into clinical workflows. The advancements includes increased access to digital imaging datasets, larger imaging data sets, increased used of imaging in healthcare and increased computing power \cite{cite03}\cite{cite05}. These factors combine to create an ideal state for research related to advanced machine learning techniques, namely artifical neural networks, and the implementation of tools that can rival, and even outperform, the assessment of highly trained radiologists.

\section{Machine Learning and Artificial Neural Networks}

Broadly speaking, machine learning is a way of applying artificial intelligence to a problem through the analysis of data. Machine learning techniques assess the features, or attributes, of samples in a dataset to identify patterns in the data, and the resulting algorithm can be used to render conclusions about new inputs without human intervention \cite{cite05}. Medical image analysis presents what is referred to as a classification problem. The typical example of a classifcation problem is handwritten numerical digit recognition. In this example, a handwritten digit between 0 and 9 is fed into the algorithm, and the algorithm decides which of the ten digits, or classes, that handwritten digit is most likely to belong. Specific to medical image analysis, the classifier detects abnormalities in images otherwise not present in images of the same area in healthy individuals, and renders a conclusion as to what that abnormality is.

There are several different machine learning techniques that have traditionally been used in classification problems, such as support vector machines (SVM). SVMs are an example of a supervised machine learning model, meaning this method uses labeled data. Every sample in the dataset includes the label, or correct answer, along with a value for each of the attributes. The SVM identifies patterns in the labeled samples to create an algorithm, and new, unlabeled samples can be processed by the algorithm and given a prediction. In the task of medical image analysis, attributes have historically been identified and designed by human experts \cite{cite06}. For example, an expert would identify abnormalities by explicitly describing shape, texture, position and orientation of the abnormal biological structure \cite{cite07}.  In addition to this being labor intensive, the image features are specific to the immediate problem being explored and cannot be expected to work well for other image types \cite{cite06}.

Artifical neural networks (ANN) are another class of machine learning techniques that are increasingly being leveraged to tackle problems related to image analysis. ANNs, just like SVMs, are often utilized in a supervised manner, but do not require the painstaking process of expert-defined key attributes. Instead, ANNs learn the important features from the images themselves \cite{cite07}. This is an obvious advantage when compared to traditional machine learning methods, but the complexity of these algorithms has prevented the achievment of mainstream use until recently. The theory of ANNs was introduced in the 1950s, and research in this field has begun to flourish recently due to the increase in computing power and availability of high quality datasets \cite{cite05}.

\section{How ANNs Work}

The inspiration for the design of ANNs is drawn from the way a brain works. 

\section{Applications in Medical Image Analysis}

There are several specialities in which medical image analysis has been studied and applied for the purpose of computer-aided detection and diagnosis. ANNs have been used in studies to compare performance of radiologists to computerized approches. A study 



Diabetic retinopathy is the third leading cause of blindness worldwide, contributing to 

\section{Infrastructure}



\section{Challenges}

There are several challenges that may inhibit the success of widespread use of CAD tools built upon a neural network algorithm. First, overfitting occurs when an algorithm is trained based on a dataset that does not generalize well to examples outside of the data used to train the algorithm. Many studies demonstrating the value of ANNs were trained using relatively small datasets, and because the significant features present in a small dataset may not be the exact same features present in a large dataset, the algorithm derived from the small dataset may not perform well when analyzing images from the large dataset \cite{cite08} \cite{cite05}. This issue can be addressed by training algorithms on larger datasets, but of course requires access to larger datasets, longer training periods and more computing power.

Second, algorithms derived from ANNs are frequently considered to be black boxes, meaning that the results are uninterpretable. This contrasts with several other types of machine learning techniques that produce equations that much more easily lend themselves to understanding which features are significant and why \cite{cite05}. Instill belief and trust in a sytem that is difficult, if not impossible, to explain is a barrier, even if that system produces accurate responses the vast majority of time.

Third, ethical and legal considerations must be made. Adopters of this technology must grapple with questions pertaining to the liability if and when the system makes a prediction that harms a patient \cite{cite05}. If a radiologist is led to a conclusion by an algorithm, and the algorithm presents a false positive, a false negative or presents one conclusion while missing another, who is responsible for the error?

Fourth, the ANNs are dependent on the quality and nature of the imaging data used to train algorithms. There is variability around the world in regards to the type and quality of imaging machines and the imaging protocols that dictate why and how images are taken \cite{cite05}. It's conceivable that two different imaging machines or two different techniques capturing two images of similar abnormalities could appear different to an algorith such that the conclusion is not the same. This issue could at least partially be addressed by sufficiently large datasets containing labeled images of abnormalities that were taken using machines of varying quality and executed using differing methods.  

\section{Conclusion}

math used - sigmoid function - why what does this do

back propagation

what is deep learning, how does it compare with human brains

difference between types on neural networks, which are best suited to the tast

 
\begin{acks}

 These are my acknowledgements

\end{acks}


\bibliographystyle{ACM-Reference-Format}
\bibliography{report} 


\appendix
 

\section{Issues}

\DONE{Example of done item: Once you fix an item, change TODO to DONE}

\subsection{Assignment Submission Issues}

    \TODO{Do not make changes to your paper during grading, when your repository should be frozen.}

\subsection{Uncaught Bibliography Errors}

    \TODO{Missing bibliography file generated by JabRef}
    \TODO{Bibtex labels cannot have any spaces, \_ or \& in it}
    \TODO{Citations in text showing as [?]: this means either your report.bib is not up-to-date or there is a spelling error in the label of the item you want to cite, either in report.bib or in report.tex}

\subsection{Formatting}

    \TODO{Incorrect number of keywords or HID and i523 not included in the keywords}
    \TODO{Other formatting issues}

\subsection{Writing Errors}

    \TODO{Errors in title, e.g. capitalization}
    \TODO{Spelling errors}
    \TODO{Are you using {\em a} and {\em the} properly?}
    \TODO{Do not use phrases such as {\em shown in the Figure below}. Instead, use {\em as shown in Figure 3}, when referring to the 3rd figure}
    \TODO{Do not use the word {\em I} instead use {\em we} even if you are the sole author}
    \TODO{Do not use the phrase {\em In this paper/report we show} instead use {\em We show}. It is not important if this is a paper or a report and does not need to be mentioned}
    \TODO{If you want to say {\em and} do not use {\em \&} but use the word {\em and}}
    \TODO{Use a space after . , : }
    \TODO{When using a section command, the section title is not written in all-caps as format does this for you}\begin{verbatim}\section{Introduction} and NOT \section{INTRODUCTION} \end{verbatim}

\subsection{Citation Issues and Plagiarism}

    \TODO{It is your responsibility to make sure no plagiarism occurs. The instructions and resources were given in the class}
    \TODO{Claims made without citations provided}
    \TODO{Need to paraphrase long quotations (whole sentences or longer)}
    \TODO{Need to quote directly cited material}

\subsection{Character Errors}

    \TODO{Erroneous use of quotation marks, i.e. use ``quotes'' , instead of " "}
    \TODO{To emphasize a word, use {\em emphasize} and not ``quote''}
    \TODO{When using the characters \& \# \% \_  put a backslash before them so that they show up correctly}
    \TODO{Pasting and copying from the Web often results in non-ASCII characters to be used in your text, please remove them and replace accordingly. This is the case for quotes, dashes and all the other special characters.}
    \TODO{If you see a figure and not a figure in text you copied from a text that has the fi combined as a single character}

\subsection{Structural Issues}

    \TODO{Acknowledgement section missing}
    \TODO{Incorrect README file}
    \TODO{In case of a class and if you do a multi-author paper, you need to add an appendix describing who did what in the paper}
    \TODO{The paper has less than 2 pages of text, i.e. excluding images, tables and figures}
    \TODO{The paper has more than 6 pages of text, i.e. excluding images, tables and figures}
    \TODO{Do not artificially inflate your paper if you are below the page limit}

\subsection{Details about the Figures and Tables}

    \TODO{Capitalization errors in referring to captions, e.g. Figure 1, Table 2}
    \TODO{Do use {\em label} and {\em ref} to automatically create figure numbers}
    \TODO{Wrong placement of figure caption. They should be on the bottom of the figure}
    \TODO{Wrong placement of table caption. They should be on the top of the table}
    \TODO{Images submitted incorrectly. They should be in native format, e.g. .graffle, .pptx, .png, .jpg}
    \TODO{Do not submit eps images. Instead, convert them to PDF}

    \TODO{The image files must be in a single directory named "images"}
    \TODO{In case there is a powerpoint in the submission, the image must be exported as PDF}
    \TODO{Make the figures large enough so we can read the details. If needed make the figure over two columns}
    \TODO{Do not worry about the figure placement if they are at a different location than you think. Figures are allowed to float. For this class, you should place all figures at the end of the report.}
    \TODO{In case you copied a figure from another paper you need to ask for copyright permission. In case of a class paper, you must include a reference to the original in the caption}
    \TODO{Remove any figure that is not referred to explicitly in the text (As shown in Figure ..)}
    \TODO{Do not use textwidth as a parameter for includegraphics}
    \TODO{Figures should be reasonably sized and often you just need to
  add columnwidth} e.g. \begin{verbatim}/includegraphics[width=\columnwidth]{images/myimage.pdf}\end{verbatim}

re


\end{document}
